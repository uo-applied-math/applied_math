\documentclass{article}
\usepackage[utf8]{inputenc}
\usepackage{amsmath, amsthm, amssymb}

\usepackage{mdframed}
\usepackage{upgreek}
\newcommand{\Hom}{\text{Hom}}
\newcommand{\Comma}{\mathbin{{,}}}
\usepackage{graphicx}
\graphicspath{ {./images/} }
\usepackage[english]{babel}
\usepackage{graphicx}
\usepackage[justification=centering]{subfig}
\usepackage{hyperref}
\usepackage{spverbatim}
\usepackage{mathtools}
\usepackage[OT2,T1]{fontenc}
\usepackage{tikz-cd}
\DeclareSymbolFont{cyrletters}{OT2}{wncyr}{m}{n}
\DeclareMathSymbol{\Sha}{\mathalpha}{cyrletters}{"58}

        
\newtheorem{theorem}{Theorem}
\newtheorem*{remark*}{Remark}
\newtheorem*{theorem*}{Theorem}
\newtheorem{corollary}{Corollary}[theorem]
\newtheorem{conjecture}{Conjecture}
\newtheorem{lemma}[theorem]{Lemma}
\newtheorem*{lemma*}{Lemma}
\newtheorem{prop}{Proposition}
\theoremstyle{definition}
\newtheorem{definition}{Definition}[section]
\newtheorem{example}{Example}
\newtheorem*{example*}{Example}

\newcommand \Q {\mathbb Q}
\newcommand \R {\mathbb R}
\newcommand \C {\mathbb C}
\newcommand \Z {\mathbb Z}

\usepackage[margin=1in]{geometry}
\DeclareMathOperator{\Tr}{Tr}

\title{607 Applied Notes}
% \author{Andy Huchala}
\date{2/3/2020}

\begin{document}

\maketitle
Erratum for homework 4, question 2: The probability that the tower \underline{does not} fall over during a given $t$ minutes is $e^{-nt}$.\\\\

\Large
\noindent \textbf{List of topics for Midterm 1:}\normalsize
\begin{enumerate}
    \item \underline{Poisson point processes:}
    \begin{itemize}
        \item definition, assumptions
        \item thinning, labeling, additivity
        \item how to simulate one
        \item exponential waiting times for $PP(\lambda \text dt)$
        \item characteristic function
    \end{itemize}
    \item \underline{General:}
    \begin{itemize}
        \item covariance
        \item independence
        \item conditional expectation
        \item computing expectation (by conditioning)
        \item law of total (co)variance
    \end{itemize}
    \item \underline{Continuous-time Markov chains:}
    \begin{itemize}
        \item definition(s):
        \begin{itemize}
            \item jump rates $\xrightarrow{} G$
            \item exponential waiting times + jump probabilities
        \end{itemize}
        \item transition probabilities $\left(e^{tG}\right)_{xy}$
        \item stationary distribution $\pi G = 0$
        \item hitting probabilities: solve $Gh = 0$
        \item hitting times: solve $Gh = -1$
    \end{itemize}
\end{enumerate}
Midterm will be 3 problems, solve 2.\\

\underline{Recall} Q\# 3) How long does an outbreak last? (on average)\\
\noindent Let $\tau = \inf\{t\geq 0 : X(t) = 0\}$. Find 
$$\mathbb E[\tau | X(0) = 1]$$

Recall
$$G = \begin{bmatrix} -\delta & \delta & 0 & 0\\ \mu & -(\mu +\lambda) & \lambda & 0\\
0 & 2\mu & -2(\mu + \lambda) & 2\lambda\\
0 & 0 & 3\mu & -3(\mu +\lambda)
\end{bmatrix}
$$

$$G_1 = \delta$$
$$G_{n,n+1} = n\lambda$$
$$G_{n,n-1} = n\mu$$
$$G_{nn} = -n(\mu + \lambda)$$

\begin{theorem*}
Let $X$ be a CTMC on $S$ with generator matrix $G$, and $A\subset S$. Define $$\mathfrak{h}_A(x) = \mathbb E[\tau_A|X(0) = x] := \mathbb E^x[\tau_A]$$
where $$\tau_a = \inf\{t\geq 0:X(t)\in A\}$$
Then $\mathfrak h_A$ is the unique solution to 
$$\mathfrak h_A(x) = 0 \,\,\,\,\,\,\, \text{if } x\in A$$
$$G\mathfrak h_A(x) = -1 \,\,\,\,\,\,\, \text{if } x \notin A
$$
\end{theorem*}
\begin{corollary}
If it makes sense,
$$\mathfrak h_A = \left(G_{-A,-A}\right)^{-1}\mathbb{ONE}
$$
where 
$$(G_{-A,-A})_{xy} = G_{xy}\,\,\,\,\text{for }x\notin A, y\notin A$$
\end{corollary}
Let $\mathfrak h_0(x) = \mathbb E^x[\tau]$. Then $\mathfrak h(0) = 0$. We want is $\mathfrak h(1) = ?$
$$\delta(\mathfrak h(1) - \mathfrak h(0)) =-1
$$
$$n\lambda (\mathfrak h(n+1) - \mathfrak h(n)) + n \mu (\mathfrak h(n) - \mathfrak h(n-1)) = -1
$$
so 
$$\mathfrak h(n+1) = \dfrac{\mu + \lambda}{\lambda}\mathfrak h(n) - \dfrac{\mu}{\lambda}\mathfrak h(n-1) - 1
$$
How to solve this? Generating functions would work. Or could just plug it in (just need first two values, then can go from there).
$$a(n) = (\mathfrak h(n+1) - \mathfrak h(n))\left(\dfrac{\lambda}{\mu}\right)^n \implies a(n) - a(n-1) = -\dfrac{1}{n\lambda} \left(\dfrac{\lambda}{\mu}\right)^n
$$
where we can take $a(n)$ to be a discrete analogue of the derivative of $\mathfrak h$ (from which we could ``integrate'' to obtain $\mathfrak h(n)$).

\begin{proof}
$$\mathfrak h_A(x) = \dfrac{1}{-G_{xx}} + \sum_{y\neq x} \dfrac{G_{xy}}{-G_{xx}}\mathfrak h_A(y)
$$
where the first term is the mean time until leaves $x$, the fraction on the right is the probability that it jumps from $x\to y$ next, and the rightmost term is the mean remaining time.
\end{proof}
\begin{proof}
We could write down proof number 2 but we're out of time.
\end{proof}
\end{document}
