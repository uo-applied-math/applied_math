\documentclass[12pt]{article}
\usepackage{amssymb, enumerate, amsmath, amsthm}
\usepackage{graphics}
\newcommand{\extr}{\overline{\mathds{R}}}
\newcommand{\suun}[1]{\textsuperscript{\underline{#1}}}
\newcommand{\lnorm}{\left\lvert \left \lvert }
\newcommand{\rnorm}{\right \rvert \right \rvert}
\newcommand{\M}{\mathcal{M}}
\newcommand{\supp}{\text{supp}}
\newcommand{\ol}[1]{\overline{#1}}
\newcommand{\im}{\text{Im}}
\newcommand{\re}{\text{Re}}

\theoremstyle{plain}
\newtheorem{theorem}{Theorem}[section]
\newtheorem{corollary}[theorem]{Corollary}
\newtheorem{lemma}[theorem]{Lemma}
\newtheorem{proposition}[theorem]{Proposition}
\newtheorem{conjecture}[theorem]{Conjecture}
\newtheorem{question}{Question}
\newtheorem*{theorem*}{Theorem}
\newtheorem*{example*}{Example}
\newtheorem*{definition*}{Definition}
\newtheorem*{nonexample*}{Non-Example}


\setlength{\oddsidemargin}{0in}
\setlength{\textwidth}{6.5in}
\setlength{\topmargin}{0in}
\setlength{\headheight}{0in}
\setlength{\headsep}{0in}
\setlength{\textheight}{8.7in}
\title{MATH 607 (Applied Math II) Lecture Notes}
\author{Christophe Dethier}
\date{January 13, 2019}
\begin{document}
\bigskip
\maketitle

Lab is in 13 Pacific, from 1:00 - 2:30pm on Tuesdays.\\

A few last details from the raindrop problem from last week. If $N \sim \text{Pois}(\lambda)$, then $\mathbb{P}\{N = 0\} = e^{-\lambda}$. One can visualize the number of drops which intersect a point $p$ as the volume of the cone above $p$ in $xyr$ space. Let's move on to some new examples.\\

Example: Let $N \sim \text{Pois}(\lambda)$, $X \sim \text{Binomial}(N,p)$, and $Y = N - X$. Then we claim that $X$ and $Y$ are independent, $X \sim \text{Pois}(\lambda)$, and $Y \sim \text{Pois}((1 - p)\lambda)$. (Note that $X$ and $Y$ are no longer independent if one conditions on $N$.) To prove this claim, we will compute the joint probability generating function. If we can show that
\[
\mathbb{E}\left[ a^X b^Y\right] = e^{\lambda p(a - 1)} e^{\lambda(1 - p)(b - 1)},
\]
then the claim will be demonstrated. To do this, we use three different properties:
\begin{enumerate}[(1)]
\item $\displaystyle \mathbb{E}\left[ u^N\right] = \sum_{n \geq 0} \frac{\lambda^n}{n!} e^{-\lambda} u^n = e^{\lambda(u - 1)}$
\item $\displaystyle \mathbb{E}\left[ u^X | N\right] = \sum_{k=1}^N {N \choose k} (1 - p)^{N - k} u^k = (1 - p + pu)^N$.
\item $\displaystyle \mathbb{E}\left[ f(X) \right] = \mathbb{E}\left[ \mathbb{E}\left[ f(X) | Y\right] \right]$. This is known as the ``tower property".
\end{enumerate}
First, we use (3) to condition on $N$;
\[
\mathbb{E}\left[ a^X b^Y\right] = \mathbb{E}\left[ a^X b^{N-X}\right] = \mathbb{E}\left[b^N \mathbb{E}\left[ \left( \frac{a}{b} \right)^X | N\right] \right].
\]
Next we use (2) to evaluate the inner expectation and (1) to evaluate the outer expectation;
\begin{align*}
\mathbb{E}\left[b^N \mathbb{E}\left[ \left( \frac{a}{b} \right)^X | N\right] \right] &= \mathbb{E}\left[ b^N\left(1 - p + p \frac{a}{b}\right)^N \right] \\
&= \exp\left(\lambda \left(b \left( 1 - p + p \frac{a}{b} \right) - 1 \right) \right) \\
&= \exp(\lambda((1 - p)(b - 1) + p ( a -1 )).
\end{align*}
So the claim is proved. \\

\text{underline:} Let $\{x_1,\ldots,x_m\} \subseteq T$, where $T = [0,1]^2$ is the torus. Are they from a PPP with constant intensity? (Constant intensity means mean measure proportional to Lebesgue measure.)

To analyze this problem, we first need a general fact about PPP's, conditional uniformity: If $N \sim \text{PPP}(\mu)$ on $X$ and $\{y_1,\ldots, y_m\}$ are points of $N$ in $A \subseteq X$, then $\{y_1,\ldots,y_m\}$ are independent, and distributed according to $Y$ given $N(A) = m$, where 
\[
\mathbb{P}\{Y \in dy\} = \frac{1}{\mu(A)} \mu(dy).
\] 


The idea here is to quantify ``clusteriness", or over/under dispersion. To do this, let $\rho: \mathbb{R}_{\geq 0} \rightarrow \mathbb{R}_{\geq 0}$ be a decreasing function with $\lim_{r \rightarrow \infty} \rho(r) = 0$ and $\int_T \rho(|x|) dx = 1$. Let
\[
P_i = \sum_{j \neq i} \rho(|x_j - x_i|), \quad \quad \text{ and } \quad \quad P = \sum_i P_i.
\]
This $P$ measures ``clusteriness" at scale $\rho$. Note that
\[
P = \int_S \rho(|x - y|) N(dx)N(dy),
\]
where $S = \{(x,y) \in T^2: x \neq y\}$. Suppose that the points are from a $\text{PPP}(\lambda dx dy)$. Let's compute $\mathbb{E}[P]$.
\[
\mathbb{E}[P] = \int_S \rho(|x - y|) \mathbb{E}[N(dx)N(dy)],
\]
where $\mathbb{E}[N(dx)N(dy)]$ is the expected pairs of points at $x$ and at $y$. That is,
\[
\frac{\mathbb{E}[N(dx)N(dy)]}{dxdy} = \lim_{\epsilon rightarrow 0} \frac{1}{|B_{\epsilon}(x)|^2} \mathbb{E}[N(B_{\epsilon}(x)) N(B_{\epsilon}(y))] = \lambda^2
\]
if $x \neq y$. So
\[
\mathbb{E}[P] = \int_S \rho(|x - y|) \lambda^2 dx dy = \lambda^2.
\]



\end{document}