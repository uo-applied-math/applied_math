\documentclass[12pt]{amsart}   
\usepackage[margin=1in]{geometry}                		
\geometry{letterpaper}   




\title{Friday Week 2 Notes}
\date{October 11, 2019}


\begin{document}
\maketitle


\section{Binary Search}

Input: \\
$L$ - a sorted list of numbers\\
$x$ - the number we're looking for\\
$a$ - place in $L$ to start looking\\
$b$ - place in $L$ to stop looking\\

Output:\\
true or false - is $x$ in the list between $a$ and $b$ or not\\

Some code for binary search: \\ 

def Find(L,x,a,b):\\
\indent \indent \indent let $c = \lfloor{\frac{a+b}{2}\rfloor}$\\
\indent \indent \indent if $a = = b$:\\
\indent \indent \indent \indent \indent \indent return $(L(a) = = x)$\\
\indent \indent \indent else:\\
\indent \indent \indent \indent \indent \indent if $L[c] < x:$\\
\indent \indent \indent \indent \indent \indent \indent \indent \indent return Find(L,x,c,b)\\
\indent \indent \indent \indent \indent \indent else:\\
\indent \indent \indent \indent \indent \indent \indent \indent \indent return Find(L,x,a,c)\\

Let's analyze how long this takes to run:\\
Let $T(n) = $ worst-case running time on list of length n\\
\underline{Claim.} $T(n) = max\left(T(\lfloor{\frac{n}{2}\rfloor}), T(\lceil{\frac{n}{2}\rceil})\right) + C$
\vspace{10mm}

\underline{Claim.} T is increasing, so $T(n) = T(\lceil{\frac{n}{2}\rceil}) + C$ 

If $n = 2^k$, 
$$T(2^k) = T(2^{k-1}) + C$$
$$ = T(2^{k-2} + 2C$$
$$...$$
$$ = T(1) + kC$$

We learn $T(n) \leq Clog_2n$ (assuming T is increasing), i.e. $T(n) = O(logn)$.


\hrulefill

\section{Memoization}

If you have something like:\\

def F(a,b,c):\\
\indent \indent \indent do some things\\
\indent \indent \indent return answer\\

and you would like this to run faster the next time you call it, you can do:\\

memoize = $\{\}$\\
def F(a,b,c):\\
\indent \indent \indent if (a,b,c) not in memoize:\\
\indent \indent \indent \indent \indent \indent do some things\\
\indent \indent \indent \indent \indent \indent memoize[(a,b,c)] = answer\\
\indent \indent \indent return memoize[(a,b,c)]\\

note that (a,b,c) have to be things you can put in a dictionary\\

\vspace{10mm}

\underline{Example.} Up-Right lattice points from $(0,0)$ to $(a,b)$\\

Let $n(a,b)$ be the number of U-R paths from $(0,0$ to $(a,b)$\\

It's clear that\\


$$ 
	n(a,b) = 
	\begin{cases}
	1 \;\; \text{if $a = 0$ or $b = 0$}\\
	n(a-1,b)+n(a,b-1) \\
	\end{cases}
$$

You have some options:\\
- could memoize\\
- could compute these for $(a,b)$ in increasing lex order\\
- answer really though is ${a+b \choose b}$, could do math\\

\vspace{10mm}

\section{Graph Theory}

\begin{itemize}
\item{A \underline{graph} $G = (V,E)$ has a set $V$ of vertices and $E \subseteq (V \times V)$ of edges.}
\item{We say G is \underline{simple} if $(v,v) \notin E$ for every $v \in V$ (i.e. no loops of vertex to itself).}
\item{We say G is \underline{directed} if $E$ is a set of ordered pairs $(a,b)$.}
\item{We say G is \underline{undirected} if $E$ is a set of unordered pairs $\{a,b\}$.}
\item{A \underline{walk} from $a \in V$ to $b \in V$ is a sequence of edges
$$(a_1,a_1), (a_1,a_2),...,(a_{k-1},a_k),(a_k,b)$$
in $E$.}
\item{A \underline{connected} graph has a walk from a to b for all $a,b \in V$.}
\item{A \underline{path} from a to b is a walk with no repeated vertex.}
\item{A \underline{cycle} is a walk from $a \in V$ to a of length $\geq 3$ with no repeated vertices except for endpoints.}
\item{A \underline{tree} is a connected graph with no cycles.}
\end{itemize}

\vspace{10mm}

\underline{Claim.} If $T = (V,E)$ is a tree, then $|V| = |E| +1$. 

\end{document}