\documentclass[12pt,letterpaper]{article}
\usepackage[latin1]{inputenc}
\usepackage[T1]{fontenc}
\usepackage{amsmath}
\usepackage{amsfonts}
\usepackage{amssymb}
\usepackage{graphicx}
\usepackage{amsthm}
\usepackage{tikz}
\usepackage{verbatim}
\theoremstyle{definition}
\newtheorem{example}{Example}[section]
\begin{document}
\section{Analysis of Algorithms}

\begin{itemize}
	\item how long your code takes to run, with "how long" referring to the number of steps. Then the question remains: what is a step?
	\item look for rough bounds, depend on size of the problem but don't depend on what computer you have
\end{itemize}

We say $f(n)$ is $O(g(n))$ if there exist constants $C$, $N$ such that $\forall n \geq N$, we have $f(n) \leq Cg(n)$. Or a lower bound $\Omega$, then $f(n) \geq Cg(n)$. 

We say $f(x)$ is $\theta(g(x))$ if both $f(x)$ is $O(g(x))$ and $f(x)$ is $\Omega(g(x))$. 

\begin{example}[Bubble Sort Algorithm]
	Input: list of numbers $L = [\ell_0, \ldots \ell_n]$. 
	
	Output: $L$, but sorted.
	Repeat the following: 
	
	\noindent\texttt{for i from 0 to n-1:}\\
	\hspace*{4ex}\texttt{if $\ell_i > \ell_{i+1}$:}\\
	\hspace*{8ex}\texttt{swap $l_i$ with $l_{i+1}$ in $L$}
	
\end{example}
	
\end{document}