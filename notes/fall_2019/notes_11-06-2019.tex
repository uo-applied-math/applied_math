\documentclass{article}

\usepackage[utf8]{inputenc}
\usepackage{amsmath}
\usepackage{amsfonts}
\usepackage{amssymb}
\usepackage[margin=.75in, total={8.25in, 10.75in}, heightrounded]{geometry}

\usepackage{graphicx}
\graphicspath{{graphics/}}

\usepackage{hyperref}

\usepackage{scalerel}

\usepackage{xcolor}

\usepackage{stmaryrd}

\usepackage{MnSymbol}

\usepackage{mdframed}

\usepackage{titlesec}

\usepackage{blkarray}

\usepackage{etex}

\titleformat{\section}
{\normalfont \Large \bfseries \centering}{\Roman{section} --- }{0pt}{}




\definecolor{DefGreen}{rgb}{0,0.5,0}
\definecolor{TheoremOrange}{rgb}{0.88,0.6,0.08}
\definecolor{LemmaYellow}{rgb}{1,1,0}
\definecolor{CorollaryBlue}{rgb}{0,0.3,1}
\definecolor{ProofPurple}{rgb}{0.58,0,1}
\definecolor{AxiomRed}{rgb}{1,0,0}
\definecolor{CommentBlue}{rgb}{0.46,0.67,1}



\usepackage{amsthm}

\newtheoremstyle{colontheorem}
	{0in}                    	% Space above
	{.15in}                   	% Space below
	{\normalfont}      		    % Body font
	{}                          % Indent amount
	{\bfseries}                 % Theorem head font
	{:}                         % Punctuation after theorem head
	{.5em}                      % Space after theorem head
	{}							% Theorem head spec (can be left empty, meaning ‘normal’)
	
\theoremstyle{colontheorem}

\newtheorem{theorem}{Theorem}[section]
\newtheorem{proposition}[theorem]{Proposition}
\newtheorem{definition}[theorem]{Definition}
\newtheorem{axiom}[theorem]{Axiom}

\newtheorem{lemma}{Lemma}[theorem]
\newtheorem{corollary}{Corollary}[theorem]

\newtheorem{exercise}{Exercise}[section]




\newcommand{\Span}{\textnormal{span}}
\newcommand{\Null}{\textnormal{null }}
\newcommand{\Range}{\textnormal{range }}
\newcommand{\T}{^\textnormal{T}}

\newcommand{\Sub}{\textnormal{sub }}

\newcommand{\re}{\textnormal{Re }}
\newcommand{\im}{\textnormal{Im }}
\newcommand{\Arg}{\textnormal{Arg }}
\newcommand{\Log}{\textnormal{Log }}
\newcommand{\Res}{\textnormal{Res}}
\newcommand{\pv}{\textnormal{p.v.}}

\newcommand{\e}{\varepsilon}




\newenvironment{Theorem}
{
	\begin{mdframed}[backgroundcolor=TheoremOrange!10]
	\begin{theorem}
}
{
	\end{theorem}
	\end{mdframed}
	
	\vspace{.15in}
}

\newenvironment{Proposition}
{
	\begin{mdframed}[backgroundcolor=TheoremOrange!10]
	\begin{proposition}
}
{
	\end{proposition}
	\end{mdframed}
	
	\vspace{.15in}
}

\newenvironment{Def}
{
	\begin{mdframed}[backgroundcolor=DefGreen!10]
	\begin{definition}
}
{
	\end{definition}
	\end{mdframed}
	
	\vspace{.15in}
}

\newenvironment{Axiom}
{
	\begin{mdframed}[backgroundcolor=AxiomRed!10]
	\begin{axiom}
}
{
	\end{axiom}
	\end{mdframed}
	
	\vspace{.15in}
}

\newenvironment{Lemma}
{
	\begin{mdframed}[backgroundcolor=LemmaYellow!10]
	\begin{lemma}
}
{
	\end{lemma}
	\end{mdframed}
	
	\vspace{.03in}
}

\newenvironment{Corollary}
{
	\begin{mdframed}[backgroundcolor=CorollaryBlue!10]
	\begin{corollary}
}
{
	\end{corollary}
	\end{mdframed}
	
	\vspace{.09in}
}

\newenvironment{Comment}
{
	\begin{mdframed}[backgroundcolor=CommentBlue!10]
	\textbf{Comment:}%
}
{
	\end{mdframed}
	
	\vspace{.15in}
}

\newenvironment{Proof}
{
	\begin{mdframed}[backgroundcolor=ProofPurple!10]
	\textbf{Proof:}%
}
{
	\end{mdframed}
	
	\vspace{.085in}
}

\newenvironment{Example}
{
	\begin{mdframed}
	\textbf{Example:}%
}
{
	\end{mdframed}
	
	\vspace{.15in}
}



\setlength{\parindent}{0pt}




\begin{document}

\vspace*{.5in}

\begin{center}
	\Huge Applied Notes\\
	
	\vspace{.25in}
	
	\Large November 6th, 2019\\
\end{center}

\vspace{.5in}



\begin{Example}
	%
	Let $K_n$ be the complete graph on $n$ vertices. Then the number of spanning trees of $K_n$ is $\det \Delta_i^i$ for any $i$. If we choose $i = n$, we have
	
	$$
		\det \Delta_i^i = \det \begin{bmatrix}
 			n - 1 & -1 & \cdots & -1\\
 			-1 & n - 1 & \cdots & -1\\
 			\vdots & \vdots & \ddots & \vdots\\
 			-1 & -1 & \cdots & n - 1
 		\end{bmatrix}.
	$$
	
	Subtracting row $1$ from each other row,
	
	$$
		\det \Delta_i^i = \det \begin{bmatrix}
 			n - 1 & -1 & -1 & \cdots & -1\\
 			-n & n & 0 & \cdots & 0\\
 			-n & 0 & n & \cdots & 0\\
 			\vdots & \vdots & \vdots & \ddots & \vdots\\
 			-n & 0 & 0 & \cdots & n
 		\end{bmatrix}.
	$$
	
	Now adding every column except column $1$ to column $1$,
	
	$$
		\det \Delta_i^i = \det \begin{bmatrix}
 			1 & -1 & -1 & \cdots & -1\\
 			0 & n & 0 & \cdots & 0\\
 			0 & 0 & n & \cdots & 0\\
 			\vdots & \vdots & \vdots & \ddots & \vdots\\
 			0 & 0 & 0 & \cdots & n
 		\end{bmatrix}.
	$$
	
	By a theorem from Cauchy, this determinant is equal to $n^{n - 2}$, so there are $n^{n - 2}$ spanning trees of $K_n$.
	
\end{Example}



\begin{Def}
	
	Let $G$ be an undirected graph and let $e = (e_0, e_1)$ and $f = (f_0, f_1)$ be arbitrarily oriented edges in $G$. Let $\overleftarrow{f} = (f_1, f_0)$. $J^e(f) = \mathbb{E}(\textnormal{\# times } f \textnormal{ is used in a random walk from } e_0 \textnormal{ to } e_1) = \mathbb{E} \left( \textnormal{\# times } \overleftarrow{f} \textnormal{ is used} \right)$. $\beta(e, f) = \mathbb{P}(\textnormal{the path from } e_0 \textnormal{ to } e_1 \textnormal{ in a uniformly random spanning tree of } G \textnormal{ uses } f)$.
	
\end{Def}



\begin{Theorem}
	
	$\beta(e, f) - \beta \left( e, \overleftarrow{f} \right) = J^e(f)$.
	
	\begin{Proof}
		%
		$\beta(e, f) - \beta \left( e, \overleftarrow{f} \right)$ is the expected number of times a loop-erased random walk from $e_0$ from $e_1$ uses $f$, minus the amount of times it uses $\overleftarrow{f}$. This is because a loop-erased random walk either uses $f$ once or not at all, so the probability is the same as the expectation. But in the non-loop-erased walk, the probability of walking through $f$ forwards is the same as the probability of walking through it backwards, so the expected number of times $f$ is used inside of a loop is the same as the expected number of times that $\overleftarrow{f}$ is. Thus $\beta(e, f) - \beta \left( e, \overleftarrow{f} \right)$ is the expected number of times $f$ is used outside of a loop in a random walk from $e_0$ to $e_1$, minus the expected number of times $\overleftarrow{f}$ is used. But this is exactly $J^e(f)$.
		
	\end{Proof}
	
\end{Theorem}



\begin{Comment}
	%
	$J^e$ describes the current flowing through a graph $G$ when a $1$V battery is placed on edge $e$ and $1\Omega$ resistors are placed on every edge.
	
\end{Comment}







\end{document}