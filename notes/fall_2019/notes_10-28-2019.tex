\documentclass[12pt]{article}
\usepackage[utf8]{inputenc}
\usepackage{amsmath, amsthm, amssymb, amsfonts}

\usepackage{upgreek}
\usepackage[margin=1in]{geometry}
\usepackage{graphicx}
\graphicspath{ {./images/} }
\usepackage[english]{babel}
\usepackage{graphicx}
\usepackage[justification=centering]{subfig}
\usepackage{hyperref}
\usepackage{spverbatim}
\usepackage{mathtools}
\usepackage[OT2,T1]{fontenc}
\usepackage[lastexercise]{exercise}
\DeclareSymbolFont{cyrletters}{OT2}{wncyr}{m}{n}
\DeclareMathSymbol{\Sha}{\mathalpha}{cyrletters}{"58}
\newcommand{\overbar}[1]{\mkern 1.5mu\overline{\mkern-1.5mu#1\mkern-1.5mu}\mkern 1.5mu}

\newtheorem{theorem}{Theorem}
\newtheorem{claim}{Claim}
\newtheorem{corollary}{Corollary}[theorem]
\newtheorem{conjecture}{Conjecture}
\newtheorem{lemma}[theorem]{Lemma}
\newtheorem{prop}{Proposition}

\newcommand \Q {\mathbb Q}
\newcommand \R {\mathbb R}
\newcommand \Z {\mathbb Z}

\let \int \undefined
\DeclareMathOperator \int {int}

\title{Applied Day 12}
% \author{Andy Huchala}
\date{October 28, 2019}
\begin{document}

\maketitle
Peter Ralph lecturing.
\section{Configuration model}
A random graph such that every degree $\sim D$ with probability given by $\mathbb P\{D=n\}= d_n$. Let $$\mu = \dfrac{\mathbb E[D(D-1)]}{\mathbb E[D]}$$
$$=\sum_{n\geq 0} \dfrac{n d_n}{\sum_{m\geq 0}md_m} (n-1)
$$be the mean of the the ``size-biased distribution of $D-1$'', ie picking a vertex with probability proportional to the number of edges connected to it. We consider the generating function of $D$ and the sized-biased version of $D$:
$$\Phi(z) = \mathbb E[z^D] = \sum_{n\geq 0} z^n d_n
$$
$$\Psi(z) = \dfrac{\mathbb E[z^{D-1}D]}{\mathbb E[D]}=\sum_{n\geq 0}z^n \dfrac{nd_n}{\sum_{m\geq 0}md_m}
$$
Recall from homework 1 that $\mu = \Psi'(1)$.

At step $t$, let $$A_t = \text{``active vertices''}\,\,\,\,(\text{where }A_0=0)$$
$$V_t = \text{``visited vertices''}\,\,\,\, (\text{note }|V_t|=t)$$
$$U_t = \text{``unvisited vertices''}\,\,\,\,[N]\setminus(A_t\cup V_t)$$
We explore by
\begin{enumerate}
    \item taking an active vertex
    \item move it to $V$
    \item adding its children\footnote{Children isn't quite the right word\ldots we want to add the vertices it's connected to which aren't already in $A$ or $V$} to $A$
\end{enumerate}
We will compare this to a branching process.

Let $N_0=1$. Want: $N_t$ to give us the size of a branching process so far.\\

\noindent Note: branching process goes generation by generation, so we defined $N_t$ slightly different.\\

\noindent Let $N_t=|A_t| + E_t+t$ where $E_t$ is the ``excess''. This is not quite a branching process since we get some cross connections, e.g. a node having muliple parents. We want it to look as much as possible like an exploration process; when we hit a trouble edge we do (something).

$$E_0 = 0$$
$$E_{t+1} = \text{\#(edges between vertices in $V_t$ except the original ones)}$$
$$=\text{\#(edges between vertices in $V)-(t-1)$}
$$

Note: \# edges in a tree with vertices is $n-1$.

\underline{Claim:} Let $(X_k)_{k\geq 0}$ be a branching process with $\mathbb P\{X_0 = n\} = d_n$ and offspring distribution with generating function $\Psi(z)$. Then
$$\left(\lim_{t\xrightarrow{}\infty}N_t\right) = N_{\infty}$$ has the same distribution as
$$\left(\lim_{k\xrightarrow{}\infty} \sum_{s=0}X_s = T_k\right) = T_{\infty}
$$ the total size of the branching process.

\begin{proof}
This is obvious (???)
\end{proof}
\begin{proof}
\begin{enumerate}
    \item Let's number each vertex $v$ with a ``generation'' $g(v)$ so that
    $$g(v) = g(\text{parent of }v)+1$$
    i.e. $g(v)$ is the min of $g(\text{parents of v})+1$
    \item Let $X_k = \#\{\text{vertices $v:g(v)=k$}\}+(\text{something else related to $E$})$
    where 
    $$(\text{something})_k = \sum_{l=0}^k Y_{l,(k-l)}
    $$
    the number of edges from generation $k$ to generation $k$.\\
    \noindent Let $B_k = \#\{\text{edges in excess connecting to generation k}\}$\\
    
    \noindent For $t \geq 0$, $0\leq j \leq B_t -1$, let $Y_{tj}$ be an independent branching process with the same distribution as $X$. Then $(\text{something})_k$ counts how many extra things were added in this new branching process.
    
    Then the idea of this proof is that in the limit the amount of excess should go to zero (?).
    
\end{enumerate}
$N_{\infty}\geq |A_{\infty} + T$ where $A_{\infty}=0$, the number of active vertices at the end of the exploration process, and $T$ is the size of component and $1=|C|$ so if $N_{\infty}$ is finite then so is $|C|$. We assume the connected component is finite.
\end{proof}

\underline{Fact:} Proportion of nodes \underline{not} in the giant component is asymptotically the probability of extinction of the branching process.

\end{document}
